\documentclass{article}
\usepackage[utf8]{inputenc}
\usepackage{dsfont}
\usepackage{amsmath}
\usepackage{mdsymbol}
\usepackage{xcolor}

% \usepackage{nath}
% \delimgrowth=1



\title{Causal Inference - Exercise 1}
\author{Amir Gavrieli}
\date{15.11.2019}

\begin{document}

\maketitle

\section*{Question 1}
Using potential outcomes notation, give an example of a data generating process (a joint distribution) which includes a hidden confounder $H$, a binary treatment $t$, and two potential outcomes $Y_0$ and $Y_1$, such that:
\begin{enumerate}
    \item Ignorability does \textit{not} hold, and
    \item $\mathds{E}[Y_1-Y_0]\neq{}\mathds{E}[Y|t=1]-\mathds{E}[Y|t=0]$, where $Y=t\cdot{}Y_1+(1-t)\cdot{}Y_0$
\end{enumerate}

\subsection*{Answer}
Consider the following setup:
	\begin{enumerate}
		\item $x  {\sim}$\textit{Ber}($q$)
		\item $T|x  {\sim}$\textit{Ber}($p_1x+p_2(1-x)$)
	\end{enumerate}
	And $Y$ is a linear function of $x,T$: $Y=4T+4x$\\ 
	W conclude that:
	\begin{itemize}
		\item $Y_0= 4x$
		\item $Y_1= 4+4x$
	\end{itemize}
	
	First let us show that ignorability does not hold:
	\begin{equation*}

	\end{equation*}
	

\end{document}